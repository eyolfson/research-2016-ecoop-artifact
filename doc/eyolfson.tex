\documentclass[a4paper,UKenglish]{lipics-v2016}
 
\usepackage{microtype}

% Commands for artifact descriptions
% Written by Camil Demetrescu and Erik Ernst
% April 8, 2014

% ARTIFACT: This entire file should be used as-is; it defines standard
% headings to be included in the artifact description, and it will be 'input'
% into the document file such that you can use the environments defined
% below

\newenvironment{scope}{\section{Scope}}{}
\newenvironment{content}{\section{Content}}{}
\newenvironment{getting}{\section{Getting the artifact} The artifact 
endorsed by the Artifact Evaluation Committee is available free of 
charge on the Dagstuhl Research Online Publication Server (DROPS).}{}
\newenvironment{platforms}{\section{Tested platforms}}{}
\newcommand{\license}[1]{{\section{License}#1}}
\newcommand{\mdsum}[1]{{\section{MD5 sum of the artifact}#1}}
\newcommand{\artifactsize}[1]{{\section{Size of the artifact}#1}}


\newcommand{\const}{{\bfseries \ttfamily const}}

% Author macros::begin %%%%%%%%%%%%%%%%%%%%%%%%%%%%%%%%%%%%%%%%%%%%%%%%%%%%%%%%%
\title{C++ \const{} and Immutability: An Empirical Study of
       Writes-Through-\const{} (Artifact)
  \footnote{This artifact is a companion of the paper: Jon Eyolfson and Patrick
            Lam, ``C++ \const{} and Immutability: An Empirical Study of
            Writes-Through-\const{}'', Proceedings of the 30th European
            Conference on Object-Oriented Programming (ECOOP 2016), Rome, Italy,
            July 2016.}}
\author[1]{Jon Eyolfson}
\author[2]{Patrick Lam}
\affil[1]{University of Waterloo\\
  Waterloo, ON, Canada\\
  \texttt{jeyolfso@uwaterloo.ca}}
\affil[2]{University of Waterloo\\
  Waterloo ON, Canada\\
  \texttt{patrick.lam@uwaterloo.ca}}
\authorrunning{J. Eyolfson and P. Lam}

\Copyright{Jonathan Eyolfson and Patrick Lam}
\subjclass{D.3.3 Language Constructs and Features}
\keywords{empirical study, dynamic analysis, immutability}
% Author macros::end %%%%%%%%%%%%%%%%%%%%%%%%%%%%%%%%%%%%%%%%%%%%%%%%%%%%%%%%%%

%Editor-only macros::begin %%%%%%%%%%%%%%%%%%%%%%%%%%%%%%%%%%%%%%%%%%%%%%%%%%%%
\EventEditors{John Q. Open and Joan R. Acces}
\EventNoEds{2}
\EventLongTitle{42nd Conference on Very Important Topics (CVIT 2016)}
\EventShortTitle{CVIT 2016}
\EventAcronym{CVIT}
\EventYear{2016}
\EventDate{December 24--27, 2016}
\EventLocation{Little Whinging, United Kingdom}
\EventLogo{}
\SeriesVolume{42}
\ArticleNo{23}
% Editor-only macros::end %%%%%%%%%%%%%%%%%%%%%%%%%%%%%%%%%%%%%%%%%%%%%%%%%%%%%

\begin{document}

\maketitle

\begin{abstract}
  This artifact is based on {\tt ConstSanitizer}, a dynamic program analysis
  tool that detects deep immutability violations through \const{} qualifiers.
  Our tool instruments any code compiled by \texttt{clang} with the
  \texttt{-fsanitizer=const} flag. Our implementation includes both
  instrumentation of LLVM code and a runtime library to support our analysis.
  The provided package includes our tool and all experiments used in our
  companion paper. Instructions are also provided.
\end{abstract}

% ARTIFACT: section on the scope of the artifact (what claims of the paper are
%           intended to be backed by this artifact?)
\begin{scope}
  This artifact includes the full source code for our tool,
  \texttt{ConstSanitizer}, and supports the repeatability of our experiments in
  the companion paper. We encourage users to reproduce our results and use the
  tool in other software projects.
\end{scope}

% ARTIFACT: section on the contents of the artifact (code, data, etc.)
\begin{content}
  The artifact package includes:
  \begin{itemize}
    \item a Virtual Machine image with everything included
    \item detailed instructions for using the artifact and for rebuilding it
          from scratch, provided as an {\tt index.html} file.
  \end{itemize}

  To simplify repeatabiliy of our experiments, we provide a QEMU and VirtualBox
  disk image containing a minimal GNU/Linux distribution with all required
  sources for the tool and experiments. The image is based on Arch Linux,
  up-to-date on 2016-04-22. For reference, the Linux kernel version is 4.4.3.
  This image includes a pre-compiled version of our tool available through the
  default \texttt{clang} and \texttt{clang++} commands. The instructions outline
  how to build our tool from scratch, if desired, however it is time consuming.
  All of the experiments, including their sources and dependencies, are
  pre-installed so an internet connection is not required.
\end{content} 

% ARTIFACT: section containing links to sites holding the latest version of the
%           code/data, if any
\begin{getting}
  The latest version of the artifact is available on GitHub:
  {\url https://github.com/eyolfson/research-2016-ecoop-artifact}. The
  \texttt{README} in this repository provides the current links to all artifact
  packages and sources. The sources linked in this \texttt{README} are the
  modified LLVM code for the tool. This repository also contains the scripts for
  running all experiments.
\end{getting} 

% ARTIFACT: section specifying the platforms on which the artifact is known to
%           work, including requirements beyond the operating system such as
%           large amounts of memory or many processor cores
\begin{platforms}
  The artifact is known to work on any platform running QEMU or VirtualBox.
  We highly recommend CPUs with hardware virtualization and at least 2~GiB of
  memory for the Virtual Machine. The scripts themselves depend on Arch Linux,
  or any other variant that has the \texttt{pacman} command.

  The we originally developed and tested the tool on a machine running Arch
  Linux, up-to-date at the time of publication. The machine is an Intel i7-3930K
  with 32~GiB of RAM, and a 256 GB SSD.
\end{platforms}

% ARTIFACT: section specifying the license under which the artifact is
% made available
\license{
  \begin{description}
    \item[LLVM Modifications.]
      University of Illinois/NCSA Open Source License\\
      ({\url https://opensource.org/licenses/UoI-NCSA.php})
    \item[Scripts.]
      GPL-3.0 ({\url http://www.gnu.org/licenses/gpl-3.0.en.html})
  \end{description}
}

% ARTIFACT: section specifying the md5 sum of the artifact master file uploaded
%           to the Dagstuhl Research Online Publication Server, enabling
%           downloaders to check that the file is the expected version and
%           suffered no damage during download.
\mdsum{
  \texttt{0d30b05ad520209348fd38a3662e45f5} (QEMU)

  \noindent \texttt{6d5a88c36de2cec6fa50bd74d601cea9} (VDI)
}

% ARTIFACT: section specifying the size of the artifact master file uploaded to
%           the Dagstuhl Research Online Publication Server
\artifactsize{1.5 GB}

\end{document}
